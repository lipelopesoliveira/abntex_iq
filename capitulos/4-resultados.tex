%% abtex2-modelo-include-comandos.tex, v-1.9.6 laurocesar
%% Copyright 2012-2016 by abnTeX2 group at http://www.abntex.net.br/ 
%%
%% This work may be distributed and/or modified under the
%% conditions of the LaTeX Project Public License, either version 1.3
%% of this license or (at your option) any later version.
%% The latest version of this license is in
%%   http://www.latex-project.org/lppl.txt
%% and version 1.3 or later is part of all distributions of LaTeX
%% version 2005/12/01 or later.
%%
%% This work has the LPPL maintenance status `maintained'.
%% 
%% The Current Maintainer of this work is the abnTeX2 team, led
%% by Lauro César Araujo. Further information are available on 
%% http://www.abntex.net.br/
%%
%% This work consists of the files abntex2-modelo-include-comandos.tex
%% and abntex2-modelo-img-marca.pdf
%%

% ---
% Este capítulo, utilizado por diferentes exemplos do abnTeX2, ilustra o uso de
% comandos do abnTeX2 e de LaTeX.
% ---

\chapter{Conteúdos específicos do modelo de trabalho acadêmico}
\label{cap_trabalho_academico}

\section{Quadros}

Este modelo vem com o ambiente \texttt{quadro} e impressão de Lista de quadros configurados por padrão. Verifique um exemplo de utilização:

\begin{quadro}[htb]
	\caption{\label{quadro_exemplo}Exemplo de quadro}
	\begin{tabular}{|c|c|c|c|}
		\hline
		\textbf{Pessoa} & \textbf{Idade} & \textbf{Peso} & \textbf{Altura} \\ \hline
		Marcos & 26    & 68   & 178    \\ \hline
		Ivone  & 22    & 57   & 162    \\ \hline
		...    & ...   & ...  & ...    \\ \hline
		Sueli  & 40    & 65   & 153    \\ \hline
	\end{tabular}
	\fonte{Autor.}
\end{quadro}

Este parágrafo apresenta como referenciar o quadro no texto, requisito obrigatório da ABNT. Primeira opção, utilizando \texttt{autoref}: Ver o \autoref{quadro_exemplo}. Segunda opção, utilizando  \texttt{ref}: Ver o Quadro \ref{quadro_exemplo}.

\section{Código}

É possível também inserir blocos de código que se comportam como uma figura, como o exemplo apresentado na \autoref{MSER}.

\begin{figure}[!ht]
	\centering
	\caption{Exemplo de código para utilização do pyMSER. A função \texttt{equilibrate()} aplica o método MSER nos dados obtidos da simulação e gera um pequeno relatório com os resultados.}
	\label{MSER}
	\begin{lstlisting}[language=Python]
		import pymser
		import pandas as pd
		
		# Load the .csv file
		df = pd.read_csv('example_data/Cu-BTT_500165.0_198.000000.csv')
		
		results = pymser.equilibrate(
		df['mol/kg'], 
		LLM=True, 
		batch_size=1, 
		ADF_test=True, 
		uncertainty='uSD', 
		print_results=True
		)
	\end{lstlisting}
	\fonte{O autor.}
	
\end{figure}


\section{Tabela}

Em geral o modelo padrão de tabela do LaTeX pode ser meio feio, a \autoref{methods} apresenta um exemplo de tabela customizada. O comando \texttt{arraystretch} pode ser alterado para ajudar a redimensionar a tabela caso os dados sejam muito grandes para a folha. 

\begin{table}[ht!]
	\caption{Comparação entre tempo de execução, ponto de truncamento e valores médios da quantidade adsorvida obtidos com os diferentes métodos de equilibração no mesmo conjunto de dados.}
	\label{methods}
	\centering
	\renewcommand{\arraystretch}{1}
	\begin{tabular}{l|c|c|c}
		\hline\hline
		\textbf{Método} & \textbf{Tempo de execução (s)} & \textbf{t\textsubscript{0}} ($\times$10\textsuperscript{7}) & \textbf{Quant. adsorvida (mol/kg)} \\ \hline
		Gowers          & 0,05   & 2,57 & 39,85 $\pm$ 0,56 \\
		pyMSER          & 0,16   & 2,63 & 39,80 $\pm$ 0,56 \\
		RCA             & 0,51   & 5,92 & 40,24 $\pm$ 0,24 \\
		pyMBAR          & 78,93  & 7,69 & 40,07 $\pm$ 0,14 \\
		\hline \hline
	\end{tabular}
\end{table}
